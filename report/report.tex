\documentclass[12pt, a4paper]{article}

\usepackage[utf8]{inputenc}
\usepackage[T1]{fontenc}
\usepackage{lmodern}
\renewcommand*\familydefault{\sfdefault}
\usepackage[serbian]{babel}
\usepackage{geometry}
\usepackage{upgreek}
\geometry{a4paper}

\title{Izveštaj}
\author{Aleksa Stević}
\date{Septembar 2023.}

\begin{document}
    \maketitle
    \newpage
    \section{Opis podataka}

    Skup podataka sadrži podatke o kvalitetu vode iz 3276 različitih vodenih tela.
    
    \begin{itemize}
        \item \textbf{PH - PH vrednost}:
            Predstavlja balans između kiselog i baznog karaktera u vodi.
            WHO preporuka za ovu vrednost je od 6.5 do 8.5.
        \item \textbf{Hardness - tvrdoća}:
            Tvrdoća vode uglavnom dolazi od kalcijumovih i magnezijumovih soli.
            Ovaj atribut pretstavlja sposobnost vode da istaloži sapun od kalcijuma i magnezijuma. (??? TODO).
        \item \textbf{Solids (TDS) - čvrsti minerali}:
            Voda ima sposobnost da rastvori veliki broj minerala kao što su: kalijum, kalcijum, natrijum, bikarbonati,
            hloridi, magnezijum, sulfati.
            Preporučena vrednost količine čvrstih minerala u vodi je između 500 mg/l i 1000 mg/l.
        \item \textbf{Chloramines - hloramini}:
            Formiraju se kada se amonijak doda hloru za tretman vode.
            Preporučene vrednosti su do 4 mg/l.
        \item \textbf{Sulphates - sulfati}:
            Prirodno prisutni u mineralima i stenama.
            Koncentracija u svežoj vodi varira između 3 i 30 mg/l.
        \item \textbf{Conductivity - provodljivost}:
            Pokazuje koliko dobro voda provodi električnu struju.
            Prema WHO standardima, vrednost električne provodljivosti ne bi trebalo da prelazi 400 $\upmu$S/cm.
        \item \textbf{Organic\_carbon - organiski ugljenik}:
            Mera ukupnog ugljenika u organskim jedinjenjima u vodi.
            Prema US EPA standardima, <2 mg/l za pitku vodu.
        \item \textbf{Trihalomethanes - trihalometani}:
            Hemikalije prisutne u vodi tretiranoj hlorom.
            Sigurne vrednosti su do 80 ppm (en. parts per million).
        \item \textbf{Turbidity - zamućenost}:
            Zamućenost je uzrokovana prisustvom čvrstih materija koje su rastvorene u vodi.
            Predstavlja meru svetlosnih svojstva vode.
            Preporučene WHO vrednosti su do 5.0 NTU.
        \item \textbf{Potability - voda za piće}:
            Ovo je ciljni atribut, pokazuje da li je voda bezbedna za piće.
            Vrednost \texttt{0} znači ``za piće'', \texttt{1} znači da nije za piće
    \end{itemize}

    \section{Odabir modela}
    \subsection{Stabla odlučivanja}
    Nzm
\end{document}